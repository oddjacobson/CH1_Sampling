% Options for packages loaded elsewhere
\PassOptionsToPackage{unicode}{hyperref}
\PassOptionsToPackage{hyphens}{url}
%
\documentclass[
]{article}
\usepackage{amsmath,amssymb}
\usepackage{iftex}
\ifPDFTeX
  \usepackage[T1]{fontenc}
  \usepackage[utf8]{inputenc}
  \usepackage{textcomp} % provide euro and other symbols
\else % if luatex or xetex
  \usepackage{unicode-math} % this also loads fontspec
  \defaultfontfeatures{Scale=MatchLowercase}
  \defaultfontfeatures[\rmfamily]{Ligatures=TeX,Scale=1}
\fi
\usepackage{lmodern}
\ifPDFTeX\else
  % xetex/luatex font selection
\fi
% Use upquote if available, for straight quotes in verbatim environments
\IfFileExists{upquote.sty}{\usepackage{upquote}}{}
\IfFileExists{microtype.sty}{% use microtype if available
  \usepackage[]{microtype}
  \UseMicrotypeSet[protrusion]{basicmath} % disable protrusion for tt fonts
}{}
\makeatletter
\@ifundefined{KOMAClassName}{% if non-KOMA class
  \IfFileExists{parskip.sty}{%
    \usepackage{parskip}
  }{% else
    \setlength{\parindent}{0pt}
    \setlength{\parskip}{6pt plus 2pt minus 1pt}}
}{% if KOMA class
  \KOMAoptions{parskip=half}}
\makeatother
\usepackage{xcolor}
\usepackage[margin=1in]{geometry}
\usepackage{color}
\usepackage{fancyvrb}
\newcommand{\VerbBar}{|}
\newcommand{\VERB}{\Verb[commandchars=\\\{\}]}
\DefineVerbatimEnvironment{Highlighting}{Verbatim}{commandchars=\\\{\}}
% Add ',fontsize=\small' for more characters per line
\usepackage{framed}
\definecolor{shadecolor}{RGB}{248,248,248}
\newenvironment{Shaded}{\begin{snugshade}}{\end{snugshade}}
\newcommand{\AlertTok}[1]{\textcolor[rgb]{0.94,0.16,0.16}{#1}}
\newcommand{\AnnotationTok}[1]{\textcolor[rgb]{0.56,0.35,0.01}{\textbf{\textit{#1}}}}
\newcommand{\AttributeTok}[1]{\textcolor[rgb]{0.13,0.29,0.53}{#1}}
\newcommand{\BaseNTok}[1]{\textcolor[rgb]{0.00,0.00,0.81}{#1}}
\newcommand{\BuiltInTok}[1]{#1}
\newcommand{\CharTok}[1]{\textcolor[rgb]{0.31,0.60,0.02}{#1}}
\newcommand{\CommentTok}[1]{\textcolor[rgb]{0.56,0.35,0.01}{\textit{#1}}}
\newcommand{\CommentVarTok}[1]{\textcolor[rgb]{0.56,0.35,0.01}{\textbf{\textit{#1}}}}
\newcommand{\ConstantTok}[1]{\textcolor[rgb]{0.56,0.35,0.01}{#1}}
\newcommand{\ControlFlowTok}[1]{\textcolor[rgb]{0.13,0.29,0.53}{\textbf{#1}}}
\newcommand{\DataTypeTok}[1]{\textcolor[rgb]{0.13,0.29,0.53}{#1}}
\newcommand{\DecValTok}[1]{\textcolor[rgb]{0.00,0.00,0.81}{#1}}
\newcommand{\DocumentationTok}[1]{\textcolor[rgb]{0.56,0.35,0.01}{\textbf{\textit{#1}}}}
\newcommand{\ErrorTok}[1]{\textcolor[rgb]{0.64,0.00,0.00}{\textbf{#1}}}
\newcommand{\ExtensionTok}[1]{#1}
\newcommand{\FloatTok}[1]{\textcolor[rgb]{0.00,0.00,0.81}{#1}}
\newcommand{\FunctionTok}[1]{\textcolor[rgb]{0.13,0.29,0.53}{\textbf{#1}}}
\newcommand{\ImportTok}[1]{#1}
\newcommand{\InformationTok}[1]{\textcolor[rgb]{0.56,0.35,0.01}{\textbf{\textit{#1}}}}
\newcommand{\KeywordTok}[1]{\textcolor[rgb]{0.13,0.29,0.53}{\textbf{#1}}}
\newcommand{\NormalTok}[1]{#1}
\newcommand{\OperatorTok}[1]{\textcolor[rgb]{0.81,0.36,0.00}{\textbf{#1}}}
\newcommand{\OtherTok}[1]{\textcolor[rgb]{0.56,0.35,0.01}{#1}}
\newcommand{\PreprocessorTok}[1]{\textcolor[rgb]{0.56,0.35,0.01}{\textit{#1}}}
\newcommand{\RegionMarkerTok}[1]{#1}
\newcommand{\SpecialCharTok}[1]{\textcolor[rgb]{0.81,0.36,0.00}{\textbf{#1}}}
\newcommand{\SpecialStringTok}[1]{\textcolor[rgb]{0.31,0.60,0.02}{#1}}
\newcommand{\StringTok}[1]{\textcolor[rgb]{0.31,0.60,0.02}{#1}}
\newcommand{\VariableTok}[1]{\textcolor[rgb]{0.00,0.00,0.00}{#1}}
\newcommand{\VerbatimStringTok}[1]{\textcolor[rgb]{0.31,0.60,0.02}{#1}}
\newcommand{\WarningTok}[1]{\textcolor[rgb]{0.56,0.35,0.01}{\textbf{\textit{#1}}}}
\usepackage{graphicx}
\makeatletter
\def\maxwidth{\ifdim\Gin@nat@width>\linewidth\linewidth\else\Gin@nat@width\fi}
\def\maxheight{\ifdim\Gin@nat@height>\textheight\textheight\else\Gin@nat@height\fi}
\makeatother
% Scale images if necessary, so that they will not overflow the page
% margins by default, and it is still possible to overwrite the defaults
% using explicit options in \includegraphics[width, height, ...]{}
\setkeys{Gin}{width=\maxwidth,height=\maxheight,keepaspectratio}
% Set default figure placement to htbp
\makeatletter
\def\fps@figure{htbp}
\makeatother
\setlength{\emergencystretch}{3em} % prevent overfull lines
\providecommand{\tightlist}{%
  \setlength{\itemsep}{0pt}\setlength{\parskip}{0pt}}
\setcounter{secnumdepth}{-\maxdimen} % remove section numbering
% definitions for citeproc citations
\NewDocumentCommand\citeproctext{}{}
\NewDocumentCommand\citeproc{mm}{%
  \begingroup\def\citeproctext{#2}\cite{#1}\endgroup}
\makeatletter
 % allow citations to break across lines
 \let\@cite@ofmt\@firstofone
 % avoid brackets around text for \cite:
 \def\@biblabel#1{}
 \def\@cite#1#2{{#1\if@tempswa , #2\fi}}
\makeatother
\newlength{\cslhangindent}
\setlength{\cslhangindent}{1.5em}
\newlength{\csllabelwidth}
\setlength{\csllabelwidth}{3em}
\newenvironment{CSLReferences}[2] % #1 hanging-indent, #2 entry-spacing
 {\begin{list}{}{%
  \setlength{\itemindent}{0pt}
  \setlength{\leftmargin}{0pt}
  \setlength{\parsep}{0pt}
  % turn on hanging indent if param 1 is 1
  \ifodd #1
   \setlength{\leftmargin}{\cslhangindent}
   \setlength{\itemindent}{-1\cslhangindent}
  \fi
  % set entry spacing
  \setlength{\itemsep}{#2\baselineskip}}}
 {\end{list}}
\usepackage{calc}
\newcommand{\CSLBlock}[1]{\hfill\break\parbox[t]{\linewidth}{\strut\ignorespaces#1\strut}}
\newcommand{\CSLLeftMargin}[1]{\parbox[t]{\csllabelwidth}{\strut#1\strut}}
\newcommand{\CSLRightInline}[1]{\parbox[t]{\linewidth - \csllabelwidth}{\strut#1\strut}}
\newcommand{\CSLIndent}[1]{\hspace{\cslhangindent}#1}
\usepackage{fvextra} \DefineVerbatimEnvironment{Highlighting}{Verbatim}{breaklines,commandchars=\\\{\}}
\usepackage{float}
\ifLuaTeX
  \usepackage{selnolig}  % disable illegal ligatures
\fi
\usepackage{bookmark}
\IfFileExists{xurl.sty}{\usepackage{xurl}}{} % add URL line breaks if available
\urlstyle{same}
\hypersetup{
  pdftitle={Appendix 1: Home Range Estimation using Autocorrelated Kernel Density Estimation},
  hidelinks,
  pdfcreator={LaTeX via pandoc}}

\title{Appendix 1: Home Range Estimation using Autocorrelated Kernel
Density Estimation}
\author{}
\date{\vspace{-2.5em}\today}

\begin{document}
\maketitle

\subsection{I. Background}\label{i.-background}

The purpose of this appendix is to detail the steps to estimating a home
range (HR) using continuous-time movement modelling and the \emph{ctmm}
package. This document can be used as a practical guide, where one can
use our practice dataset or one's own data to walk-through the
analytical process.

Accounting for autocorrelation is important so that we avoid biases in
our results. However, it requires some additional steps compared to most
conventional estimators, which is why we describe the process and
provide an example workflow. We strongly recommend going through the
\emph{ctmm} vignettes (see
\url{https://ctmm-initiative.github.io/ctmm/index.html}) for a more
detailed review.

Generating a home range estimate from movement data using
continuous-time movement modelling involves three main steps: 1)
variogram inspection, 2) model fitting and selection, and 3)
Autocorrelated Kernel Density Estimation (AKDE). This process can either
be done using the \emph{ctmm} package in the R environment for
statistical computing (R Core Team 2022), or using the \emph{ctmmweb}
point-and-click graphical user interface (Calabrese et al. 2021), which
streamlines the modelling steps, helping users conduct home range
analysis without the need to know the R programming language. We
describe the process using R below:

The first step is to load the necessary packages and prepare the data

\begin{Shaded}
\begin{Highlighting}[]
\CommentTok{\# you can install ctmm from CRAN, but better to get the development version for recent updates}
\NormalTok{devtools}\SpecialCharTok{::}\FunctionTok{install\_github}\NormalTok{(}\StringTok{"ctmm{-}initiative/ctmm"}\NormalTok{)}

\CommentTok{\# load packages}
\FunctionTok{library}\NormalTok{(tidyverse)}
\FunctionTok{library}\NormalTok{(ctmm)}
\end{Highlighting}
\end{Shaded}

\subsection{II. Prepare Data}\label{ii.-prepare-data}

The data must have the same format as the following dataframe with the
same column names. These are the same format required by
\emph{Movebank}. Either you can manually edit the dataframe and then
convert to a telemetry object, or put data on Movebank and import from
there, which will automatically put the data in the correct format.

Note: \texttt{individual.local.identifier} (ILI) specifies the unique ID
(usually individual or group) that you want the home range estimate for.
At the bottom of the document, we also include some example code of how
to do all of the below analysis in a single step for a list of several
ILIs.

In our study, the ILI indicated the different sampling regimes. ``All''
was the ILI for the complete segments. For this walkthrough, we will use
the data from the complete segment of SP group, and change the ILI to
the group.

\begin{Shaded}
\begin{Highlighting}[]
\CommentTok{\# read in data frame }
\CommentTok{\# filtering data from SP group, and taking the data from the complete segment {-} denoted as "all"}
\NormalTok{DATA }\OtherTok{\textless{}{-}} \FunctionTok{read.csv}\NormalTok{(}\StringTok{"../Data/CH1\_GPS\_data.csv"}\NormalTok{, }\AttributeTok{row.names =} \ConstantTok{NULL}\NormalTok{) }\SpecialCharTok{\%\textgreater{}\%} 
  \FunctionTok{filter}\NormalTok{(group }\SpecialCharTok{==} \StringTok{"SP"} \SpecialCharTok{\&}\NormalTok{ individual.local.identifier }\SpecialCharTok{==} \StringTok{"all"}\NormalTok{) }\SpecialCharTok{\%\textgreater{}\%}   \CommentTok{\#select the complete segment from SP group (could pick any group)}
\NormalTok{  dplyr}\SpecialCharTok{::}\FunctionTok{select}\NormalTok{(}\SpecialCharTok{{-}}\NormalTok{individual.local.identifier) }\SpecialCharTok{\%\textgreater{}\%} \CommentTok{\# remove prev ILI column}
  \FunctionTok{rename}\NormalTok{(}\AttributeTok{individual.local.identifier =}\NormalTok{ group) }\CommentTok{\# make group the new ILI}
\end{Highlighting}
\end{Shaded}

The data should should look like this (these are the first six rows):

\begin{Shaded}
\begin{Highlighting}[]
\FunctionTok{head}\NormalTok{(DATA)}
\end{Highlighting}
\end{Shaded}

\begin{verbatim}
##   individual.local.identifier           timestamp location.long location.lat
## 1                          SP 2010-09-12 05:00:00     -125.3790     5.500800
## 2                          SP 2010-09-12 05:30:00     -125.3791     5.500413
## 3                          SP 2010-09-12 06:00:00     -125.3793     5.499555
## 4                          SP 2010-09-12 06:30:00     -125.3795     5.498766
## 5                          SP 2010-09-12 07:00:00     -125.3792     5.497902
## 6                          SP 2010-09-12 07:30:00     -125.3786     5.497841
\end{verbatim}

Once the dataframe is in the correct format, convert it to a telemetry
object and specify the UTM projection:

\begin{Shaded}
\begin{Highlighting}[]
\CommentTok{\# convert data to telemetry object}
\NormalTok{DATA }\OtherTok{\textless{}{-}}\NormalTok{ DATA }\SpecialCharTok{\%\textgreater{}\%} 
  \FunctionTok{as.telemetry}\NormalTok{(}\AttributeTok{projection =} \StringTok{"+proj=utm +zone=10 +north +datum=WGS84 +units=m +no\_defs +ellps=WGS84 +towgs84=0,0,0"}\NormalTok{)}
\end{Highlighting}
\end{Shaded}

Plot the data:

\begin{Shaded}
\begin{Highlighting}[]
\CommentTok{\# plot location data}
\FunctionTok{plot}\NormalTok{(DATA, }\AttributeTok{main =} \StringTok{"Location Data"}\NormalTok{)}
\end{Highlighting}
\end{Shaded}

\begin{figure}[H]

{\centering \includegraphics[width=0.75\linewidth,]{../Figures/Example_Data} 

}

\caption{Location data from the complete segment of SP group}\label{fig:unnamed-chunk-10}
\end{figure}

\subsection{III. Variogram Inspection}\label{iii.-variogram-inspection}

Variograms plot the semi-variance (y-axis), which is a measure of the
average squared displacement, as a function of the time-lag that
separates any pair of observed locations (Diggle and Ribeiro 2007; Silva
et al. 2021). Variograms play two major roles in the \emph{ctmm}
workflow: first, they provide an unbiased visual diagnostic to assess
the autocorrelation structure present in the data, and second, they
inform whether the data shows evidence of range residency (Silva et al.
2021). Asymptoting curves in a variogram indicate range residency. Where
the asymptote aligns with the x-axis is a measure of the necessary
time-lag between positions to assume independence (Silva et al. 2021).
It is also a rough estimate of the home range crossing time (Christen H.
Fleming and Calabrese 2017). If the curve continues to increase without
flattening, the animals are either non-resident (i.e.~home range drift
or migration), or not tracked long enough to capture the full extent of
their home range (Calabrese, Fleming, and Gurarie 2016).

Once the data are confirmed to represented range-restricted movement, we
can proceeded with model fitting and selection. It is necessary to
confirm range-residency before conducting home range estimation because,
while ctmm is capable of fitting both range-resident (the default) and
endlessly diffusing movement models, only the first set are appropriate
for home range estimation.

\begin{Shaded}
\begin{Highlighting}[]
\NormalTok{SVF }\OtherTok{\textless{}{-}} \FunctionTok{variogram}\NormalTok{(DATA, }\AttributeTok{dt =} \FunctionTok{c}\NormalTok{(}\DecValTok{1}\NormalTok{,}\DecValTok{10}\NormalTok{) }\SpecialCharTok{\%\#\%} \StringTok{"hour"}\NormalTok{) }\CommentTok{\# dt argument changes the width of the time{-}lag bins (makes variogram smoother)}
\FunctionTok{plot}\NormalTok{(SVF, }\AttributeTok{main =} \StringTok{"Empirical Variogram"}\NormalTok{)}
\end{Highlighting}
\end{Shaded}

\begin{figure}[H]

{\centering \includegraphics[width=0.75\linewidth,]{../Figures/Example_SVF} 

}

\caption{Empirical variogram from the complete segment of SP group}\label{fig:unnamed-chunk-13}
\end{figure}

This is a plot of the empricial varigram. The line asymptotes at
approximately three days. This is roughly the average home range
crossing time. This is also approximately how far apart locations need
to be in time for them to be independent
(\url{https://ctmm-initiative.github.io/ctmm/articles/variogram.html}).

\subsection{IV. Accounting for Error}\label{iv.-accounting-for-error}

There are several options for accounting for error depending on whether
calibration data (e.g.~DOP, HDOP, etc.) exists for the user's GPS device
or not. If these data exists, follow the vignette here:
\url{https://ctmm-initiative.github.io/ctmm/articles/error.html} to
learn how to model and calibrate these errors.

Here, we describe how to handle the errors when no calibration data
exists. To account for errors without calibration data, we use a prior
for the root-mean-square User Equivalent Range Error (RMS UERE) (C.
Fleming et al. 2020). According to the \texttt{ctmm} developers, a 10-15
meter error radius is generally sufficient to account for location error
with GPS data. We assign a mean prior of 20 meters with reasonable
credible intervals to account for additional error that could be caused
by group spread or variation in researcher position with regard to the
center of the group.

\begin{Shaded}
\begin{Highlighting}[]
\DocumentationTok{\#\# FITTING UNCALIBRATED DATA UNDER A PRIOR}

\CommentTok{\# assign 20{-}meter RMS UERE}
\FunctionTok{uere}\NormalTok{(DATA) }\OtherTok{\textless{}{-}} \DecValTok{20}

\CommentTok{\# change uncertainty }
\NormalTok{UERE }\OtherTok{\textless{}{-}} \FunctionTok{uere}\NormalTok{(DATA)}
\FunctionTok{summary}\NormalTok{(UERE) }\CommentTok{\# original credible interval is zero}
\end{Highlighting}
\end{Shaded}

\begin{verbatim}
## , , horizontal
## 
##     low est high
## all  20  20   20
\end{verbatim}

\begin{Shaded}
\begin{Highlighting}[]
\CommentTok{\# give DOF a smaller value (default is INF which produces no credible intervals)}
\NormalTok{UERE}\SpecialCharTok{$}\NormalTok{DOF[] }\OtherTok{\textless{}{-}} \DecValTok{2}

\CommentTok{\# now there are reasonable credible intervals}
\FunctionTok{summary}\NormalTok{(UERE)}
\end{Highlighting}
\end{Shaded}

\begin{verbatim}
## , , horizontal
## 
##          low est     high
## all 6.960018  20 33.38156
\end{verbatim}

\begin{Shaded}
\begin{Highlighting}[]
\CommentTok{\# assign the prior to the data}
\FunctionTok{uere}\NormalTok{(DATA) }\OtherTok{\textless{}{-}}\NormalTok{ UERE}
\end{Highlighting}
\end{Shaded}

\subsection{V. Movement Model
Selection}\label{v.-movement-model-selection}

Model fitting also involves two steps: first the \texttt{ctmm.guess}
function uses the shape of the empirical variogram (and error
information) to generate starting values required for the non-linear
models, and second, the \texttt{ctmm.select} function uses the values
calculated from \texttt{ctmm.guess} to fit a range of alternative
stationary (and range-restricted) movement models using Maximum
Likelihood (Christen H. Fleming et al. 2014). Models are ranked by AICc
(Akaike information criterion) allowing us to evaluate which model or
models best predict the data. This process permits identification and
fit of a stationary movement model that corresponds to the observed
movement behavior of the animal (Christen H. Fleming et al. 2014).

It is worth noting that here, stationary means that the underlying
movement processes are assumed to be consistent throughout the duration
of the data. Movement model parameters represent time-averaged values,
which has important implications on how data should be segmented for
home range analysis. If the underlying parameters change drastically
within the sample---particularly the mean location---then the stationary
assumption has been violated. Therefore, it is common practice to
segment the data when the parameters change and estimate separate
ranges. This is consistent with Burt's original concept of the home
range where, for example, he stated winter and summer ranges for
migratory species should be considered separately with the travel
between as transit (Burt 1943). In our case, all sampling regimes were
from single, stationary ranges which negated any need for further
segmentation.

The pool of potential movement models which involve home range behavior
include:

\begin{enumerate}
\def\labelenumi{\arabic{enumi})}
\item
  \emph{Independent and Identically Distributed} (IID) -- location data
  has uncorrelated positions and velocities.
\item
  \emph{Ornstein--Uhlenbeck} (OU) -- location data has autocorrelated
  positions and uncorrelated velocities.
\item
  \emph{OU Foraging} (OUF) -- location data has autocorrelated positions
  and velocities (Calabrese, Fleming, and Gurarie 2016; C. Fleming et
  al. 2014; Christen H. Fleming et al. 2014)
\end{enumerate}

OU and OUF can be further specified with isotropic or anisotropic
versions of each. Isotropic means diffusion is equal on every extent of
the home range, while anisotropic means diffusion is asymmetrical (Silva
et al. 2021).

Endlessly diffusing movement models (non-HR models) such as
\emph{brownian motion} (BM) or \emph{integrated OU} (IOU) cannot be
statistically compared to HR models using maximum likelihood (see
\texttt{?ctmm.select}). To fit these movement models, one must manually
specify them.

\begin{Shaded}
\begin{Highlighting}[]
\CommentTok{\# get starting values for models}
\NormalTok{GUESS }\OtherTok{\textless{}{-}} \FunctionTok{ctmm.guess}\NormalTok{(DATA,}\AttributeTok{interactive=}\ConstantTok{FALSE}\NormalTok{, }
                    \AttributeTok{CTMM=}\FunctionTok{ctmm}\NormalTok{(}\AttributeTok{error=}\ConstantTok{TRUE}\NormalTok{), }\CommentTok{\# important for including error information}
                    \AttributeTok{variogram =}\NormalTok{ SVF) }

\CommentTok{\# fit models and select top one, trace = 2 allows you to see progress}
\NormalTok{FIT }\OtherTok{\textless{}{-}} \FunctionTok{ctmm.select}\NormalTok{(DATA,GUESS,}\AttributeTok{trace=}\DecValTok{2}\NormalTok{)}
\end{Highlighting}
\end{Shaded}

\begin{Shaded}
\begin{Highlighting}[]
\CommentTok{\# see model summary for top model}
\FunctionTok{summary}\NormalTok{(FIT)}
\end{Highlighting}
\end{Shaded}

\begin{verbatim}
## $name
## [1] "OUF anisotropic error"
## 
## $DOF
##      mean      area diffusion     speed 
##  53.47721  86.17602 251.31469 400.79579 
## 
## $CI
##                                low       est      high
## area (square kilometers)  2.108771  2.635754  3.220627
## τ[position] (hours)       9.226870 12.031009 15.687352
## τ[velocity] (minutes)    26.346700 31.798288 38.377904
## speed (kilometers/day)    4.791723  5.038385  5.284885
## diffusion (hectares/day) 42.700768 48.513870 54.692545
## error all (meters)       13.481921 17.779238 22.068097
\end{verbatim}

The top model selected for our practice dataset was \emph{OUF
anisotropic}. Above is the summary information for that model. The
\texttt{\$DOF} specifies the effective sample sizes. The most important
one for home range estimation being under \texttt{area} which indicates
the number of statistically independent points (or approximately the
number of home range crossings -- see Methods in the main text).

The \texttt{area\ (square\ kilometers)} slot indicates the Gaussian
area, which is an estimate of spatial variance, but is not our AKDE
area. \texttt{tau{[}position{]}\ (hours)} is the tau referenced in the
main text. This is time necessary between locations for them to be
independent, or approximately the home range crossing timescale.
\texttt{tau{[}velocity{]}\ (minutes)} is the timescale necessary for the
velocities to be independent. Estimates of speed (i.e.~proportional to
average daily travel distance) and diffusion rate are also included.

\begin{Shaded}
\begin{Highlighting}[]
\CommentTok{\# plot empirical variogram with best model}
\FunctionTok{plot}\NormalTok{(SVF, }\AttributeTok{CTMM =}\NormalTok{ FIT, }\AttributeTok{main =} \StringTok{"Variogram and Fitted Model"}\NormalTok{)}
\end{Highlighting}
\end{Shaded}

\begin{figure}[H]

{\centering \includegraphics[width=0.75\linewidth,]{../Figures/Example_SVF_model} 

}

\caption{Top movement model fitted to the empirical variogram}\label{fig:unnamed-chunk-21}
\end{figure}

\subsection{VI. AKDE Home Range
Estimation}\label{vi.-akde-home-range-estimation}

The final step is to calculate an autocorrelated kernel density
home-range estimate (AKDE) using the eponymously named \texttt{akde}
function (Calabrese, Fleming, and Gurarie 2016). This function takes the
movement data and the corresponding fitted model and returns: a
utilization distribution (UD) object corresponding to the range
distribution, information on the optimal bandwidth, point estimates and
confidence intervals for HR area, and a measure of the effective sample
size of the data for home range estimation. For the sampling regimes in
our study, we also included the \texttt{weights\ =\ TRUE} option, which
helps correct for irregular and missing data by down-weighting
over-sampled portions of the data and up-weighting under-sampled
portions (C. H. Fleming et al. 2018). This helps to offset sampling
bias, but is not sufficient if large portions of the true range are
missing from the sampled data.

\begin{Shaded}
\begin{Highlighting}[]
\CommentTok{\# get UD using AKDE}
\NormalTok{UD }\OtherTok{\textless{}{-}} \FunctionTok{akde}\NormalTok{(DATA, FIT, }\AttributeTok{weights =} \ConstantTok{TRUE}\NormalTok{)}

\CommentTok{\# plot UD over location data}
\FunctionTok{plot}\NormalTok{(DATA,}\AttributeTok{UD =}\NormalTok{ UD, }\AttributeTok{main =} \StringTok{"AKDE Home Range Estimate"}\NormalTok{)}
\end{Highlighting}
\end{Shaded}

\begin{figure}[H]

{\centering \includegraphics[width=0.75\linewidth,]{../Figures/Example_AKDE} 

}

\caption{Home range estimate (95\% utilization distribution) mean contour and 95\% confidence intervals plotted over location data}\label{fig:unnamed-chunk-24}
\end{figure}

Below is the summary information:

\begin{Shaded}
\begin{Highlighting}[]
\FunctionTok{summary}\NormalTok{(UD)}
\end{Highlighting}
\end{Shaded}

\begin{verbatim}
## $DOF
##      area bandwidth 
##  86.17602 154.42060 
## 
## $CI
##                               low      est     high
## area (square kilometers) 1.779804 2.224578 2.718211
## 
## attr(,"class")
## [1] "area"
\end{verbatim}

The effective sample size (DOF area) is the same as from the fitted
model. In this case, there was about 81 observed home range crossings in
the data. The \texttt{area\ (square\ kilometers)} slot shows the
estimated home range area and 95\% confidence intervals.

\subsection{VII. Bulk analysis for several individuals or
groups}\label{vii.-bulk-analysis-for-several-individuals-or-groups}

Below is code to demonstrate how to perform the above analysis with
multiple individuals or groups in one step using a loop (may take a
little while to run, \textasciitilde20min).

\begin{Shaded}
\begin{Highlighting}[]
\DocumentationTok{\#\# PREPARE DATA }
\CommentTok{\# take only complete segments (all)}
\CommentTok{\# make group the new individual.local.identifer (ILI)}
\CommentTok{\# change to tele object}
\NormalTok{DATA\_bulk }\OtherTok{\textless{}{-}} \FunctionTok{read.csv}\NormalTok{(}\StringTok{"Data/CH1\_GPS\_data.csv"}\NormalTok{, }\AttributeTok{row.names =} \ConstantTok{NULL}\NormalTok{) }\SpecialCharTok{\%\textgreater{}\%} 
  \FunctionTok{filter}\NormalTok{(individual.local.identifier }\SpecialCharTok{==} \StringTok{"all"}\NormalTok{) }\SpecialCharTok{\%\textgreater{}\%} \CommentTok{\# take only complete segments}
\NormalTok{  dplyr}\SpecialCharTok{::}\FunctionTok{select}\NormalTok{(}\SpecialCharTok{{-}}\NormalTok{individual.local.identifier) }\SpecialCharTok{\%\textgreater{}\%} \CommentTok{\# remove prev ILI column}
  \FunctionTok{rename}\NormalTok{(}\AttributeTok{individual.local.identifier =}\NormalTok{ group) }\SpecialCharTok{\%\textgreater{}\%} \CommentTok{\# make group the new ILI}
  \FunctionTok{as.telemetry}\NormalTok{(}\AttributeTok{projection =} \StringTok{"+proj=utm +zone=10 +north +datum=WGS84 +units=m +no\_defs +ellps=WGS84 +towgs84=0,0,0"}\NormalTok{)}

\CommentTok{\# note: when there are multiple individual.local.identifiers, as.telemetry makes a list, with each}
\CommentTok{\# individual.local.identifier being an element in the list}
\CommentTok{\# variograms, model fits, and UDs follow the same list format}

\CommentTok{\# assign UERE error info}
\FunctionTok{uere}\NormalTok{(DATA\_bulk) }\OtherTok{\textless{}{-}}\NormalTok{ UERE }\CommentTok{\# using same prior as earlier}

\CommentTok{\# make empty lists to be filled by below loop}
\NormalTok{UDs }\OtherTok{\textless{}{-}}\NormalTok{ FITs }\OtherTok{\textless{}{-}}\NormalTok{ SVFs }\OtherTok{\textless{}{-}} \FunctionTok{list}\NormalTok{()}

\DocumentationTok{\#\# BULK CALCULATIONS}
\CommentTok{\# for every ILI, make a variogram (SVF), get starter values (GUESS), select model (FIT), and calculate AKDE (UD)}
\CommentTok{\# grid argument in akde aligns UDs so that overlap function is possible if desired later on}
\ControlFlowTok{for}\NormalTok{(i }\ControlFlowTok{in} \DecValTok{1}\SpecialCharTok{:}\FunctionTok{length}\NormalTok{(DATA\_bulk))\{}
\NormalTok{  SVFs[[i]] }\OtherTok{\textless{}{-}} \FunctionTok{variogram}\NormalTok{(DATA\_bulk[[i]])}
\NormalTok{  GUESS }\OtherTok{\textless{}{-}} \FunctionTok{ctmm.guess}\NormalTok{(DATA\_bulk[[i]],}
                      \AttributeTok{CTMM=}\FunctionTok{ctmm}\NormalTok{(}\AttributeTok{error=}\ConstantTok{TRUE}\NormalTok{),}
                      \AttributeTok{interactive=}\ConstantTok{FALSE}\NormalTok{, }
                      \AttributeTok{variogram =}\NormalTok{ SVFs[[i]])}
\NormalTok{  FITs[[i]] }\OtherTok{\textless{}{-}} \FunctionTok{ctmm.select}\NormalTok{(DATA\_bulk[[i]],}
\NormalTok{                           GUESS,}
                           \AttributeTok{trace=}\DecValTok{2}\NormalTok{)}
\NormalTok{  UDs[[i]] }\OtherTok{\textless{}{-}} \FunctionTok{akde}\NormalTok{(DATA\_bulk[[i]],}
\NormalTok{                   FITs[[i]], }
                   \AttributeTok{weights =} \ConstantTok{TRUE}\NormalTok{, }
                   \AttributeTok{grid=}\FunctionTok{list}\NormalTok{(}\AttributeTok{dr=}\DecValTok{10}\NormalTok{, }\AttributeTok{align.to.origin=}\ConstantTok{TRUE}\NormalTok{)) }
\NormalTok{\}}

\CommentTok{\# make names of variograms, fits, and UDs the same as data}
\FunctionTok{names}\NormalTok{(UDs) }\OtherTok{\textless{}{-}} \FunctionTok{names}\NormalTok{(FITs) }\OtherTok{\textless{}{-}} \FunctionTok{names}\NormalTok{(SVFs) }\OtherTok{\textless{}{-}} \FunctionTok{names}\NormalTok{(DATA\_bulk)}
\end{Highlighting}
\end{Shaded}

\begin{Shaded}
\begin{Highlighting}[]
\CommentTok{\# make color blind pallette}
\NormalTok{colorblind\_pal }\OtherTok{\textless{}{-}} \FunctionTok{c}\NormalTok{(}\StringTok{"\#E69F00"}\NormalTok{, }\StringTok{"\#56B4E9"}\NormalTok{, }\StringTok{"\#009E73"}\NormalTok{, }\StringTok{"\#F0E442"}\NormalTok{, }\StringTok{"\#D55E00"}\NormalTok{, }\StringTok{"\#CC79A7"}\NormalTok{)}

\CommentTok{\# plot together}
\FunctionTok{plot}\NormalTok{(UDs, }\AttributeTok{col.DF=}\NormalTok{colorblind\_pal)}
\end{Highlighting}
\end{Shaded}

\begin{figure}[H]

{\centering \includegraphics[width=0.75\linewidth,]{../Figures/Example_AKDE_bulk_together} 

}

\caption{Home range estimate (95\% utilization distribution) mean contours and 95\% confidence intervals from the six complete segments}\label{fig:unnamed-chunk-31}
\end{figure}

\begin{Shaded}
\begin{Highlighting}[]
\CommentTok{\# plot seperate}
\FunctionTok{par}\NormalTok{(}\AttributeTok{mfrow =} \FunctionTok{c}\NormalTok{(}\DecValTok{2}\NormalTok{,}\DecValTok{3}\NormalTok{))}
\ControlFlowTok{for}\NormalTok{(i }\ControlFlowTok{in} \DecValTok{1}\SpecialCharTok{:}\FunctionTok{length}\NormalTok{(UDs))\{}
  \FunctionTok{plot}\NormalTok{(DATA\_bulk[[i]], UDs[[i]], }
       \AttributeTok{col =}\NormalTok{ colorblind\_pal[[i]], }
       \AttributeTok{col.DF=}\NormalTok{colorblind\_pal[[i]], }
       \AttributeTok{main =} \FunctionTok{names}\NormalTok{(UDs[i]))}
\NormalTok{\}}
\end{Highlighting}
\end{Shaded}

\begin{figure}[H]

{\centering \includegraphics[width=0.75\linewidth,]{../Figures/Example_AKDE_bulk_seperate} 

}

\caption{Home range estimate (95\% utilization distribution) contours and 95\% confidence intervals from the six complete segments plotted separately}\label{fig:unnamed-chunk-34}
\end{figure}

You can compare home range areas using:

\begin{Shaded}
\begin{Highlighting}[]
\FunctionTok{meta}\NormalTok{(UDs, }\AttributeTok{variable =} \StringTok{"area"}\NormalTok{, }\AttributeTok{main =} \StringTok{"HR areas"} \AttributeTok{col =}\NormalTok{ colorblind\_pal)}
\end{Highlighting}
\end{Shaded}

\begin{figure}[H]

{\centering \includegraphics[width=0.75\linewidth,]{../Figures/Example_meta} 

}

\caption{Plot showing the comparison of home range area and confidence intervals for all six complete segments. The colors correspond to the same colors in the home range plots. The mean area across the six complete segments is shown in black on the bottom}\label{fig:unnamed-chunk-38}
\end{figure}

\phantomsection\label{refs}
\begin{CSLReferences}{1}{0}
\bibitem[\citeproctext]{ref-burt43}
Burt, William Henry. 1943. {``Territoriality and Home Range Concepts as
Applied to Mammals.''} \emph{Journal of Mammalogy} 24 (3): 346--52.
\url{https://doi.org/10.2307/1374834}.

\bibitem[\citeproctext]{ref-calabrese_etal16}
Calabrese, Justin M., Chris H. Fleming, and Eliezer Gurarie. 2016.
{``Ctmm: An r Package for Analyzing Animal Relocation Data as a
Continuous-Time Stochastic Process.''} \emph{Methods in Ecology and
Evolution} 7 (9): 1124--32.
https://doi.org/\url{https://doi.org/10.1111/2041-210X.12559}.

\bibitem[\citeproctext]{ref-calabrese_etal21}
Calabrese, Justin M., Christen H. Fleming, Michael J. Noonan, and
Xianghui Dong. 2021. {``Ctmmweb: A Graphical User Interface for
Autocorrelation-Informed Home Range Estimation.''} \emph{Wildlife
Society Bulletin} 45 (1): 162--69.
\url{https://doi.org/10.1002/wsb.1154}.

\bibitem[\citeproctext]{ref-diggle_ribeiro07}
Diggle, Peter, and Paulo J. Ribeiro. 2007. \emph{Model-Based
Geostatistics}. Springer Series in Statistics. New York, NY: Springer.

\bibitem[\citeproctext]{ref-fleming_etal18}
Fleming, C. H., D. Sheldon, W. F. Fagan, P. Leimgruber, T. Mueller, D.
Nandintsetseg, M. J. Noonan, et al. 2018. {``Correcting for Missing and
Irregular Data in Home-Range Estimation.''} \emph{Ecological
Applications} 28 (4): 1003--10. \url{https://doi.org/10.1002/eap.1704}.

\bibitem[\citeproctext]{ref-fleming_calabrese17}
Fleming, Christen H., and Justin M. Calabrese. 2017. {``A New Kernel
Density Estimator for Accurate Home-Range and Species-Range Area
Estimation.''} \emph{Methods in Ecology and Evolution} 8 (5): 571--79.
\url{https://doi.org/10.1111/2041-210X.12673}.

\bibitem[\citeproctext]{ref-fleming_etal14}
Fleming, Christen H., Justin M. Calabrese, Thomas Mueller, Kirk A.
Olson, Peter Leimgruber, and William F. Fagan. 2014. {``Non-Markovian
Maximum Likelihood Estimation of Autocorrelated Movement Processes.''}
\emph{Methods in Ecology and Evolution} 5 (5): 462--72.
\url{https://doi.org/10.1111/2041-210X.12176}.

\bibitem[\citeproctext]{ref-fleming_etal14a}
Fleming, Christen, Justin Calabrese, Thomas Mueller, Kirk Olson, Peter
Leimgruber, and William Fagan. 2014. {``From Fine-Scale Foraging to Home
Ranges: A Semivariance Approach to Identifying Movement Modes Across
Spatiotemporal Scales.''} \emph{The American Naturalist} 183 (May):
E154--67. \url{https://doi.org/10.1086/675504}.

\bibitem[\citeproctext]{ref-fleming_etal20}
Fleming, Christen, Jonathan Drescher-Lehman, Michael Noonan, Tom Akre,
Donald Brown, Madaline Cochrane, Dejid Nandintsetseg, et al. 2020.
\emph{A Comprehensive Framework for Handling Location Error in Animal
Tracking Data}. \url{https://doi.org/10.1101/2020.06.12.130195}.

\bibitem[\citeproctext]{ref-rcoreteam22}
R Core Team. 2022. {``R: A Language and Environment for Statistical
Computing.''} Vienna, Austria: R Foundation for Statistical Computing.
\url{https://www.R-project.org/}.

\bibitem[\citeproctext]{ref-silva_etal21}
Silva, Inês, Christen H. Fleming, Michael J. Noonan, Jesse Alston, Cody
Folta, William Fagan, and Justin M. Calabrese. 2021.
{``Autocorrelation-Informed Home Range Estimation: A Review and
Practical Guide.''} Preprint. EcoEvoRxiv.
\url{https://doi.org/10.32942/osf.io/23wq7}.

\end{CSLReferences}

\end{document}
